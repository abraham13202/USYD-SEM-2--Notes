%% STAT5003 Comprehensive Exam Cheatsheet
%% Computational Statistical Methods
%% University of Sydney

\documentclass[10pt,landscape,a4paper]{article}
\usepackage[utf8]{inputenc}
\usepackage[T1]{fontenc}
\usepackage{multicol}
\usepackage{calc}
\usepackage{ifthen}
\usepackage[landscape]{geometry}
\usepackage{amsmath,amsthm,amsfonts,amssymb}
\usepackage{color,graphicx,overpic}
\usepackage{hyperref}
\usepackage{enumitem}
\usepackage{upquote}
\usepackage{booktabs}
\usepackage{array}
\usepackage{longtable}
\usepackage{xcolor}
\usepackage{listings}
\usepackage{tcolorbox}
\usepackage{tikz}

% Page geometry
\geometry{top=0.5cm,left=0.5cm,right=0.5cm,bottom=0.5cm}

% Redefine section commands
\makeatletter
\renewcommand{\section}{\@startsection{section}{1}{0mm}%
                                {-1ex plus -.5ex minus -.2ex}%
                                {0.5ex plus .2ex}%x
                                {\normalfont\large\bfseries\color{blue!70!black}}}
\renewcommand{\subsection}{\@startsection{subsection}{2}{0mm}%
                                {-1explus -.5ex minus -.2ex}%
                                {0.5ex plus .2ex}%
                                {\normalfont\normalsize\bfseries\color{blue!50!black}}}
\renewcommand{\subsubsection}{\@startsection{subsubsection}{3}{0mm}%
                                {-1ex plus -.5ex minus -.2ex}%
                                {1ex plus .2ex}%
                                {\normalfont\small\bfseries}}
\makeatother

% Compact lists
\setlist{nolistsep}
\setlength{\parindent}{0pt}
\setlength{\parskip}{0pt plus 0.5ex}

% Custom colors
\definecolor{codebg}{rgb}{0.95,0.95,0.95}
\definecolor{myblue}{RGB}{51,102,187}
\definecolor{myred}{RGB}{220,50,47}
\definecolor{mygreen}{RGB}{0,150,80}

% Code formatting
\lstset{
    basicstyle=\ttfamily\tiny,
    breaklines=true,
    backgroundcolor=\color{codebg},
    frame=single,
    xleftmargin=2pt,
    xrightmargin=2pt,
}

% Custom box for important formulas
\newtcolorbox{formulabox}[1][]{
  colback=blue!5!white,
  colframe=blue!75!black,
  fonttitle=\bfseries,
  title=#1,
  boxrule=0.5pt,
  arc=2pt,
  left=2pt,
  right=2pt,
  top=2pt,
  bottom=2pt
}

% Custom box for warnings/tips
\newtcolorbox{tipbox}[1][]{
  colback=red!5!white,
  colframe=red!75!black,
  fonttitle=\bfseries,
  title=#1,
  boxrule=0.5pt,
  arc=2pt,
  left=2pt,
  right=2pt,
  top=2pt,
  bottom=2pt
}

\begin{document}

\raggedright
\footnotesize

\begin{center}
     \Large{\textbf{STAT5003 Comprehensive Exam Cheatsheet}} \\
     \small{Computational Statistical Methods | University of Sydney | Final Exam 2025}
\end{center}

\begin{multicols}{3}
\setlength{\premulticols}{1pt}
\setlength{\postmulticols}{1pt}
\setlength{\multicolsep}{1pt}
\setlength{\columnsep}{2pt}

% ============================================================
\section{Performance Metrics}
% ============================================================

\subsection{Regression Metrics}

\begin{formulabox}[Key Formulas]
\textbf{MSE:} $\frac{1}{n}\sum_{i=1}^{n}(y_i - \hat{y}_i)^2$ \\
\textbf{RSS:} $\sum_{i=1}^{n}(y_i - \hat{y}_i)^2$ \\
\textbf{R²:} $1 - \frac{\sum(y_i - \hat{y}_i)^2}{\sum(y_i - \bar{y})^2}$ \\
\textbf{Adj. R²:} $1 - \frac{(1-R^2)(n-1)}{n-p-1}$
\end{formulabox}

\textbf{Key Points:}
\begin{itemize}[leftmargin=*]
    \item R² always increases with more predictors
    \item Use Adjusted R² for model comparison
    \item MSE is scale-dependent
    \item Lower RSS $\neq$ better model (overfitting)
\end{itemize}

\subsection{Classification Metrics}

\begin{formulabox}[Confusion Matrix]
\begin{tabular}{l|cc}
 & Pred Neg & Pred Pos \\ \hline
Act Neg & TN & FP \\
Act Pos & FN & TP
\end{tabular}
\end{formulabox}

\begin{formulabox}[Essential Formulas]
\textbf{Accuracy:} $\frac{TP + TN}{Total}$ \\[2pt]
\textbf{Precision:} $\frac{TP}{TP + FP}$ \\[2pt]
\textbf{Recall/Sensitivity:} $\frac{TP}{TP + FN}$ \\[2pt]
\textbf{Specificity:} $\frac{TN}{TN + FP}$ \\[2pt]
\textbf{F1 Score:} $\frac{2 \times Precision \times Recall}{Precision + Recall}$ \\[2pt]
\textbf{Cohen's Kappa:} $\frac{p_o - p_e}{1 - p_e}$
\end{formulabox}

\textbf{Cohen's Kappa Interpretation:}
\begin{itemize}[leftmargin=*]
    \item $< 0$: No agreement
    \item $0.0-0.20$: Slight
    \item $0.21-0.40$: Fair
    \item $0.41-0.60$: Moderate
    \item $0.61-0.80$: Substantial
    \item $0.81-1.00$: Almost perfect
\end{itemize}

\textbf{ROC \& AUC:}
\begin{itemize}[leftmargin=*]
    \item AUC = 0.5: Random (useless)
    \item AUC = 0.5-0.7: Poor
    \item AUC = 0.7-0.8: Acceptable
    \item AUC = 0.8-0.9: Good
    \item AUC $> 0.9$: Excellent
\end{itemize}

\subsection{Imbalanced Data}

\begin{tipbox}[CRITICAL: When Accuracy Misleads]
Example: 1\% disease prevalence \\
Predict "no disease" for all $\Rightarrow$ 99\% accuracy! \\
\textbf{Better metrics:} Precision, Recall, F1, AUC, Kappa
\end{tipbox}

\textbf{Solutions:}
\begin{enumerate}[leftmargin=*]
    \item Use F1, AUC, Cohen's Kappa
    \item Oversample minority class
    \item Undersample majority class
    \item SMOTE (synthetic samples)
    \item Adjust classification threshold
    \item Use class weights
\end{enumerate}

% ============================================================
\section{Regression Models}
% ============================================================

\subsection{Linear Regression}

\begin{formulabox}[Model]
$Y = \beta_0 + \beta_1X_1 + \cdots + \beta_pX_p + \varepsilon$
\end{formulabox}

\textbf{Objective:} Minimize $\sum_{i=1}^{n}(y_i - \hat{y}_i)^2$

\textbf{R Code:}
\begin{lstlisting}[language=R]
lm(y ~ x1 + x2 + x3, data=df)
\end{lstlisting}

\textbf{Assumptions:} Linearity, Independence, Homoscedasticity, Normality

\textbf{Pros:} Simple, interpretable, fast \\
\textbf{Cons:} Linear only, sensitive to outliers, overfits

\subsection{Ridge Regression (L2)}

\begin{formulabox}[Objective]
$\min \sum_{i=1}^{n}(y_i - \hat{y}_i)^2 + \lambda\sum_{j=1}^{p}\beta_j^2$
\end{formulabox}

\textbf{Constraint:} $\sum_{j=1}^{p}\beta_j^2 \leq s$

\textbf{R Code:}
\begin{lstlisting}[language=R]
library(glmnet)
glmnet(X, y, alpha=0, lambda=value)
\end{lstlisting}

\textbf{Key Points:}
\begin{itemize}[leftmargin=*]
    \item Shrinks coefficients toward zero (never exactly zero)
    \item All features retained
    \item Reduces variance at cost of bias
    \item Good when all predictors useful
\end{itemize}

\subsection{Lasso Regression (L1)}

\begin{formulabox}[Objective]
$\min \sum_{i=1}^{n}(y_i - \hat{y}_i)^2 + \lambda\sum_{j=1}^{p}|\beta_j|$
\end{formulabox}

\textbf{Constraint:} $\sum_{j=1}^{p}|\beta_j| \leq s$

\textbf{R Code:}
\begin{lstlisting}[language=R]
glmnet(X, y, alpha=1, lambda=value)
\end{lstlisting}

\textbf{Key Points:}
\begin{itemize}[leftmargin=*]
    \item Sets some coefficients \textbf{exactly to zero}
    \item \textbf{Feature selection} as bonus
    \item Sparse models
    \item Better with many irrelevant features
\end{itemize}

\subsection{Ridge vs Lasso}

\begin{tabular}{|p{2.5cm}|p{2cm}|p{2cm}|}
\hline
\textbf{Feature} & \textbf{Ridge} & \textbf{Lasso} \\ \hline
Penalty & L2 (squared) & L1 (absolute) \\ \hline
Feature select & No & Yes \\ \hline
Coefficients & $\to 0$ & $= 0$ \\ \hline
Geometry & Circle & Diamond \\ \hline
When to use & All useful & Many irrelevant \\ \hline
\end{tabular}

\subsection{Feature Selection}

\textbf{Methods:}
\begin{enumerate}[leftmargin=*]
    \item Best Subset: Try all $2^p$ combinations
    \item Forward: Start empty, add best
    \item Backward: Start full, remove worst
\end{enumerate}

\textbf{Selection Criteria:}
\begin{itemize}[leftmargin=*]
    \item Adjusted R² (higher better)
    \item AIC: $2p - 2\ln(L)$ (lower better)
    \item BIC: $\ln(n)p - 2\ln(L)$ (lower better)
\end{itemize}

% ============================================================
\section{Classification Models}
% ============================================================

\subsection{Logistic Regression}

\begin{formulabox}[Model]
$P(Y=1|X) = \frac{1}{1+e^{-X\beta}}$ \\[4pt]
$\log\left(\frac{p}{1-p}\right) = X\beta$
\end{formulabox}

\textbf{R Code:}
\begin{lstlisting}[language=R]
glm(y ~ x1+x2, data=df, family=binomial)
\end{lstlisting}

\textbf{Pros:} Interpretable, probabilistic, works well \\
\textbf{Cons:} Linear log-odds, can't capture complex patterns

\subsection{LDA (Linear Discriminant Analysis)}

\begin{formulabox}[Bayes Formula]
$P(Y=k|X) = \frac{\pi_k f_k(X)}{\sum_{l=1}^{K}\pi_l f_l(X)}$
\end{formulabox}

\textbf{Assumptions:}
\begin{itemize}[leftmargin=*]
    \item Normal distribution within classes
    \item \textbf{Same covariance matrix} across classes
    \item Different means
\end{itemize}

\textbf{R Code:} \texttt{lda(y \textasciitilde{} x1+x2, data=df)}

\subsection{QDA (Quadratic Discriminant Analysis)}

\textbf{Key Difference:} Each class has \textbf{different covariance}

\textbf{Decision Boundary:} Quadratic (curved)

\textbf{R Code:} \texttt{qda(y \textasciitilde{} x1+x2, data=df)}

\textbf{Parameters (p features, K classes):}
\begin{itemize}[leftmargin=*]
    \item LDA: $Kp + p(p+1)/2$
    \item QDA: $Kp + Kp(p+1)/2$ (more flexible)
\end{itemize}

\subsection{Model Comparison Table}

\begin{tabular}{|p{2cm}|p{1.5cm}|p{2cm}|}
\hline
\textbf{Method} & \textbf{Boundary} & \textbf{Assumes} \\ \hline
Logistic & Linear & None on X \\ \hline
LDA & Linear & Normal, same $\Sigma$ \\ \hline
QDA & Quadratic & Normal, diff $\Sigma$ \\ \hline
\end{tabular}

\subsection{k-Nearest Neighbors (kNN)}

\textbf{Algorithm:}
\begin{enumerate}[leftmargin=*]
    \item Choose k
    \item Find k closest training points
    \item Majority vote (classification) or average (regression)
\end{enumerate}

\textbf{R Code:} \texttt{knn(train, test, cl, k=5)}

\begin{tipbox}[Key Parameter: k]
\textbf{Small k:} Low bias, high variance (overfitting) \\
\textbf{Large k:} High bias, low variance (underfitting)
\end{tipbox}

\textbf{Important:} 
\begin{itemize}[leftmargin=*]
    \item Lazy learner (no training phase)
    \item Non-parametric
    \item \textbf{Sensitive to scale $\Rightarrow$ STANDARDIZE}
\end{itemize}

\textbf{Pros:} Simple, non-linear, no training \\
\textbf{Cons:} Slow prediction, scale-sensitive

\subsection{Support Vector Machines (SVM)}

\textbf{Objective:} Find hyperplane maximizing margin

\textbf{Key Concepts:}
\begin{itemize}[leftmargin=*]
    \item \textbf{Support vectors:} Closest points to boundary
    \item \textbf{Margin:} Distance to nearest point
    \item \textbf{C parameter:} Trade-off control
    \begin{itemize}
        \item Large C: Hard margin (low bias, high var)
        \item Small C: Soft margin (high bias, low var)
    \end{itemize}
\end{itemize}

\textbf{Kernels:}
\begin{itemize}[leftmargin=*]
    \item Linear: $K(x,x') = x^Tx'$
    \item Polynomial: $K(x,x') = (1+x^Tx')^d$
    \item RBF: $K(x,x') = \exp(-\gamma\|x-x'\|^2)$
\end{itemize}

\textbf{R Code:}
\begin{lstlisting}[language=R]
svm(y~., data=train, kernel="radial", 
    cost=C, gamma=g)
\end{lstlisting}

% ============================================================
\section{Tree-Based Methods}
% ============================================================

\subsection{Decision Trees}

\textbf{Algorithm:}
\begin{enumerate}[leftmargin=*]
    \item Find best split (max info gain)
    \item Recursively split
    \item Stop at depth/min samples
\end{enumerate}

\textbf{Splitting Criteria:}
\begin{itemize}[leftmargin=*]
    \item \textbf{Classification:} Gini or Entropy
    \begin{itemize}
        \item Gini: $\sum_{k=1}^{K}p_k(1-p_k)$
    \end{itemize}
    \item \textbf{Regression:} Minimize $\sum(y_i - \bar{y}_{node})^2$
\end{itemize}

\textbf{R Code:} \texttt{rpart(y \textasciitilde{} ., data=df, method="class")}

\textbf{Pros:} Interpretable, handles non-linear, no scaling \\
\textbf{Cons:} High variance, overfits, greedy

\subsection{Bagging (Bootstrap Aggregating)}

\begin{formulabox}[Formula]
$\hat{f}_{bag}(x) = \frac{1}{B}\sum_{b=1}^{B}\hat{f}_b^*(x)$
\end{formulabox}

\textbf{Algorithm:}
\begin{enumerate}[leftmargin=*]
    \item Create B bootstrap samples
    \item Train tree on each
    \item Average (regression) or vote (classification)
\end{enumerate}

\textbf{Out-of-Bag (OOB) Error:}
\begin{itemize}[leftmargin=*]
    \item Each bootstrap: $\sim$63\% data used
    \item Remaining $\sim$37\% for validation
    \item OOB error $\approx$ test error estimate
\end{itemize}

\textbf{R Code:}
\begin{lstlisting}[language=R]
randomForest(y~., data=df, mtry=p)
# mtry=p (all features) for bagging
\end{lstlisting}

\textbf{Key:} Reduces variance, doesn't increase bias

\subsection{Random Forest}

\textbf{Difference from Bagging:} At each split:
\begin{itemize}[leftmargin=*]
    \item Randomly sample $m$ features (typically $m = \sqrt{p}$)
    \item Choose best split from these $m$
    \item Decorrelates trees $\Rightarrow$ better
\end{itemize}

\textbf{R Code:}
\begin{lstlisting}[language=R]
randomForest(y~., data=df, 
    mtry=sqrt(p), ntree=500)
\end{lstlisting}

\textbf{Hyperparameters:}
\begin{itemize}[leftmargin=*]
    \item \texttt{ntree}: More is better (diminishing returns)
    \item \texttt{mtry}: $\sqrt{p}$ (class), $p/3$ (regression)
    \item Max depth, min samples per leaf
\end{itemize}

\textbf{Variable Importance:} Error increase when permuted

\textbf{Pros:} Very accurate, handles high-D, OOB error \\
\textbf{Cons:} Less interpretable, slower

\subsection{Boosting}

\textbf{Core Idea:} Build trees sequentially, each correcting errors

\textbf{Gradient Boosting:}
\begin{enumerate}[leftmargin=*]
    \item Fit tree to data
    \item Fit next tree to \textbf{residuals}
    \item Add: $\hat{f}(x) = \sum_{b=1}^{B}\lambda\hat{f}_b(x)$
\end{enumerate}

\textbf{Key Hyperparameters:}
\begin{itemize}[leftmargin=*]
    \item \textbf{B/n\_estimators:} Number of trees
    \item \textbf{$\lambda$/learning\_rate:} Shrinkage (0.01-0.1)
    \item \textbf{max\_depth:} Tree complexity (3-6 typical)
    \item \textbf{subsample:} Fraction of data per tree
\end{itemize}

\textbf{R Code:}
\begin{lstlisting}[language=R]
library(gbm)
gbm(y~., data=df, n.trees=100, 
    interaction.depth=3, shrinkage=0.01)
\end{lstlisting}

\textbf{Pros:} Extremely powerful, often best \\
\textbf{Cons:} Easy to overfit, needs careful tuning

\subsection{Bagging vs Boosting Comparison}

\begin{tabular}{|p{2.5cm}|p{2cm}|p{2cm}|}
\hline
\textbf{Aspect} & \textbf{Bagging/RF} & \textbf{Boosting} \\ \hline
Tree building & Parallel & Sequential \\ \hline
Sampling & Bootstrap & Weighted \\ \hline
Goal & $\downarrow$ Variance & $\downarrow$ Bias+Var \\ \hline
Overfit risk & Low & High \\ \hline
Test error & OOB & Needs CV \\ \hline
Learning rate & N/A & Critical \\ \hline
\end{tabular}

% ============================================================
\section{Dimensionality Reduction}
% ============================================================

\subsection{PCA (Principal Component Analysis)}

\textbf{Objective:} Find directions of maximum variance

\textbf{Algorithm:}
\begin{enumerate}[leftmargin=*]
    \item Center data (subtract mean)
    \item Compute covariance matrix
    \item Find eigenvectors (PCs)
    \item Project onto top k components
\end{enumerate}

\textbf{R Code:}
\begin{lstlisting}[language=R]
pca <- prcomp(X, center=TRUE, scale=TRUE)
summary(pca)  # Variance explained
pca$rotation  # Loadings
pca$x         # Scores
\end{lstlisting}

\textbf{Key Concepts:}
\begin{itemize}[leftmargin=*]
    \item \textbf{Loadings:} Feature contribution to PC
    \item \textbf{Scores:} New coordinates
    \item \textbf{Variance explained:} \% per PC
    \item \textbf{Cumulative:} Sum up to PC k
\end{itemize}

\textbf{Choosing \# PCs:}
\begin{itemize}[leftmargin=*]
    \item Scree plot (elbow)
    \item Cumulative variance $> 80-90\%$
    \item CV of downstream model
\end{itemize}

\textbf{Pros:} Preserves variance, orthogonal, interpretable \\
\textbf{Cons:} Linear only, scale-dependent

\subsection{t-SNE}

\textbf{Objective:} Preserve local structure (neighbors stay close)

\textbf{Key Parameter:} Perplexity (5-50, $\sim$neighbors)

\textbf{R Code:}
\begin{lstlisting}[language=R]
library(Rtsne)
tsne <- Rtsne(X, dims=2, perplexity=30)
plot(tsne$Y)
\end{lstlisting}

\textbf{Pros:} Excellent viz, captures non-linear, preserves clusters \\
\textbf{Cons:} Non-deterministic, can't transform new data, slow

\subsection{PCA vs t-SNE}

\begin{tabular}{|p{3cm}|p{1.2cm}|p{1.2cm}|}
\hline
\textbf{Use Case} & \textbf{PCA} & \textbf{t-SNE} \\ \hline
Feature reduction & \checkmark & $\times$ \\ \hline
Visualization & \checkmark & \checkmark \checkmark \\ \hline
Global structure & \checkmark & $\times$ \\ \hline
Local structure & $\times$ & \checkmark \\ \hline
Interpretability & \checkmark & $\times$ \\ \hline
Transform new data & \checkmark & $\times$ \\ \hline
Speed & Fast & Slow \\ \hline
\end{tabular}

\begin{tipbox}[Exam Tip]
Modeling $\Rightarrow$ PCA \\
Visualization only $\Rightarrow$ t-SNE
\end{tipbox}

% ============================================================
\section{Clustering}
% ============================================================

\subsection{K-Means}

\begin{formulabox}[Objective]
$\min \sum_{k=1}^{K}\sum_{i \in C_k}\|x_i - \mu_k\|^2$
\end{formulabox}

\textbf{Algorithm:}
\begin{enumerate}[leftmargin=*]
    \item Choose k, initialize centroids
    \item Repeat:
    \begin{itemize}
        \item Assign points to nearest centroid
        \item Update centroids to mean
    \end{itemize}
\end{enumerate}

\textbf{R Code:}
\begin{lstlisting}[language=R]
kmeans(X, centers=k, nstart=25)
\end{lstlisting}

\textbf{Choosing k:}
\begin{itemize}[leftmargin=*]
    \item Elbow method (within-SS vs k)
    \item Silhouette score
    \item Gap statistic
\end{itemize}

\textbf{Pros:} Simple, fast, scales well \\
\textbf{Cons:} Must specify k, spherical clusters, sensitive to init/outliers/scale

\subsection{Hierarchical Clustering}

\textbf{Algorithm:}
\begin{enumerate}[leftmargin=*]
    \item Each point = cluster
    \item Merge closest two clusters
    \item Repeat until one cluster
    \item Cut dendrogram at height
\end{enumerate}

\textbf{Linkage Methods:}
\begin{itemize}[leftmargin=*]
    \item \textbf{Complete:} Max distance
    \item \textbf{Single:} Min distance
    \item \textbf{Average:} Average distance
    \item \textbf{Ward's:} Min within-cluster variance
\end{itemize}

\textbf{R Code:}
\begin{lstlisting}[language=R]
dist_mat <- dist(X)
hc <- hclust(dist_mat, method="complete")
plot(hc)
clusters <- cutree(hc, k=3)
\end{lstlisting}

\textbf{Pros:} Don't need k upfront, dendrogram shows structure \\
\textbf{Cons:} Slow O($n^2$), can't undo merges

\subsection{K-Means vs Hierarchical}

\begin{tabular}{|p{2.5cm}|p{2cm}|p{2cm}|}
\hline
\textbf{Aspect} & \textbf{K-Means} & \textbf{Hierarchical} \\ \hline
Specify k & Before & After (cut) \\ \hline
Speed & Fast O(n) & Slow O($n^2$) \\ \hline
Large data & Good & Struggles \\ \hline
Deterministic & No (init) & Yes \\ \hline
Shapes & Spherical & Any \\ \hline
Visual & Scatter & Dendrogram \\ \hline
\end{tabular}

% ============================================================
\section{Resampling Methods}
% ============================================================

\subsection{Cross-Validation}

\textbf{k-Fold CV:}
\begin{enumerate}[leftmargin=*]
    \item Split data into k equal folds
    \item For each fold i:
    \begin{itemize}
        \item Train on k-1 folds
        \item Test on fold i
    \end{itemize}
    \item Average k errors
\end{enumerate}

\begin{formulabox}[CV Error]
$CV_{(k)} = \frac{1}{k}\sum_{i=1}^{k}MSE_i$
\end{formulabox}

\textbf{R Code:}
\begin{lstlisting}[language=R]
library(caret)
trainControl(method="cv", number=10)
\end{lstlisting}

\textbf{Common k:} 5 or 10 (most common), n (LOOCV)

\textbf{Bias-Variance Trade-off:}
\begin{itemize}[leftmargin=*]
    \item Small k (5): Higher bias, lower variance, faster
    \item Large k (n): Lower bias, higher variance, slower
    \item \textbf{k=10 is sweet spot}
\end{itemize}

\textbf{Repeated CV:}
\begin{lstlisting}[language=R]
trainControl(method="repeatedcv", 
             number=10, repeats=5)
\end{lstlisting}

\subsection{Bootstrap}

\textbf{Algorithm:}
\begin{enumerate}[leftmargin=*]
    \item Sample n obs. with replacement
    \item Repeat B times (100-1000)
    \item Estimate statistic on each
    \item Use distribution for inference
\end{enumerate}

\begin{formulabox}[Standard Error]
$\hat{SE}(\hat{\theta}) = \sqrt{\frac{1}{B-1}\sum_{b=1}^{B}(\hat{\theta}^{*b} - \bar{\theta}^*)^2}$
\end{formulabox}

\textbf{R Code:}
\begin{lstlisting}[language=R]
library(boot)
boot_stat <- function(data, indices) {
  return(mean(data[indices]))
}
boot(data=my_data, statistic=boot_stat, 
     R=1000)
\end{lstlisting}

\textbf{Key:} $\sim$63.2\% data used per sample, $\sim$36.8\% for validation

\textbf{Pros:} Works for any statistic, no assumptions \\
\textbf{Cons:} Can be biased for test error, computationally intensive

\subsection{CV vs Bootstrap}

\begin{tabular}{|p{2.5cm}|p{2cm}|p{2cm}|}
\hline
\textbf{Aspect} & \textbf{CV} & \textbf{Bootstrap} \\ \hline
Purpose & Model eval & Uncertainty \\ \hline
Test error & Accurate & Slight bias \\ \hline
Variance est. & No & Yes \\ \hline
Cost & Medium & High \\ \hline
Model select & Yes & Caution \\ \hline
\end{tabular}

% ============================================================
\section{Statistical Methods}
% ============================================================

\subsection{Maximum Likelihood Estimation (MLE)}

\textbf{Concept:} Find $\theta$ that maximizes P(data|$\theta$)

\begin{formulabox}[Likelihood]
$L(\theta|x) = \prod_{i=1}^{n}f(x_i|\theta)$ \\[4pt]
$\ell(\theta|x) = \sum_{i=1}^{n}\log f(x_i|\theta)$
\end{formulabox}

\textbf{Process:}
\begin{enumerate}[leftmargin=*]
    \item Write likelihood
    \item Take log
    \item Differentiate w.r.t. $\theta$
    \item Set to zero, solve
\end{enumerate}

\textbf{Example (Normal):} 
$\hat{\mu} = \bar{x}$, $\hat{\sigma}^2 = \frac{1}{n}\sum(x_i-\bar{x})^2$

\subsection{Kernel Density Estimation (KDE)}

\begin{formulabox}[Formula]
$\hat{f}(x) = \frac{1}{nh}\sum_{i=1}^{n}K\left(\frac{x-X_i}{h}\right)$
\end{formulabox}

\textbf{Kernel Properties:}
\begin{itemize}[leftmargin=*]
    \item $K(x) \geq 0$ (non-negative)
    \item $K(-x) = K(x)$ (symmetric)
    \item $\int K(x)dx = 1$ (unit measure)
\end{itemize}

\textbf{Bandwidth h:}
\begin{itemize}[leftmargin=*]
    \item Small: Undersmooth (high var, low bias)
    \item Large: Oversmooth (low var, high bias)
    \item Rule: $h = 1.06\hat{\sigma}n^{-1/5}$
\end{itemize}

\textbf{R Code:}
\begin{lstlisting}[language=R]
density(data, bw="nrd0", kernel="gaussian")
\end{lstlisting}

% ============================================================
\section{Simulation Methods}
% ============================================================

\subsection{Monte Carlo}

\textbf{Purpose:} Estimate expectations via simulation

\begin{formulabox}[Core Formula]
$E[g(X)] = \int g(x)f(x)dx \approx \frac{1}{N}\sum_{i=1}^{N}g(X_i)$
\end{formulabox}

\textbf{Algorithm:}
\begin{enumerate}[leftmargin=*]
    \item Sample $X_1, \ldots, X_N$ from f(x)
    \item Compute $g(X_i)$ for each
    \item Average: $\hat{E}[g(X)]$
\end{enumerate}

\textbf{Sampling Methods:}

\textbf{1. Inverse Transform:}
$X = F^{-1}(U)$ where $U \sim Uniform(0,1)$

\textbf{2. Acceptance-Rejection:}
\begin{enumerate}[leftmargin=*]
    \item Sample x from easy q(x)
    \item Sample u $\sim$ Uniform(0,1)
    \item If $u < f(x)/(M \cdot q(x))$, accept
    \item Else reject, repeat
\end{enumerate}

\textbf{Pseudocode Template:}
\begin{lstlisting}[language=R]
n_sims <- 10000
results <- numeric(n_sims)

for (i in 1:n_sims) {
  # Generate sample
  x <- rnorm(n, mean=mu, sd=sigma)
  # Compute statistic
  results[i] <- mean(x)
}

estimate <- mean(results)
ci <- quantile(results, c(0.025, 0.975))
\end{lstlisting}

\subsection{MCMC (Markov Chain Monte Carlo)}

\textbf{Purpose:} Sample from complex distributions

\textbf{Key Idea:} Create Markov chain with target as stationary distribution

\textbf{Metropolis-Hastings:}
\begin{enumerate}[leftmargin=*]
    \item Start with $x_0$
    \item For i = 1 to N:
    \begin{itemize}
        \item Propose $x^*$ from $q(x^*|x_{i-1})$
        \item Compute: $\alpha = \min\left(1, \frac{f(x^*)q(x_{i-1}|x^*)}{f(x_{i-1})q(x^*|x_{i-1})}\right)$
        \item With prob $\alpha$: $x_i = x^*$, else $x_i = x_{i-1}$
    \end{itemize}
\end{enumerate}

\textbf{Pseudocode:}
\begin{lstlisting}[language=R]
target <- function(x) { dnorm(x,0,1) }
proposal_sd <- 1
n_iter <- 10000
samples <- numeric(n_iter)
samples[1] <- 0

for (i in 2:n_iter) {
  proposal <- samples[i-1] + 
              rnorm(1, 0, proposal_sd)
  acceptance_ratio <- target(proposal) / 
                      target(samples[i-1])
  if (runif(1) < acceptance_ratio) {
    samples[i] <- proposal
  } else {
    samples[i] <- samples[i-1]
  }
}
samples_final <- samples[1001:n_iter]
\end{lstlisting}

\textbf{Important:}
\begin{itemize}[leftmargin=*]
    \item \textbf{Burn-in:} Discard initial samples
    \item \textbf{Thinning:} Keep every kth
    \item Check convergence (trace plots)
\end{itemize}

% ============================================================
\section{Bias-Variance Trade-off}
% ============================================================

\begin{formulabox}[Fundamental Equation]
\textbf{Total Error} = Bias$^2$ + Variance + Irreducible Error
\end{formulabox}

\textbf{Bias:} Error from wrong assumptions (underfitting) \\
\textbf{Variance:} Error from sensitivity to data (overfitting)

\subsection{Model Complexity vs Error}

\begin{center}
\textbf{Low Complexity} $\Rightarrow$ High Bias, Low Variance \\
\textbf{High Complexity} $\Rightarrow$ Low Bias, High Variance
\end{center}

\subsection{High Bias (Underfitting)}

\textbf{Signs:}
\begin{itemize}[leftmargin=*]
    \item High training error
    \item High test error
    \item Training $\approx$ test error
\end{itemize}

\textbf{Solutions:}
\begin{itemize}[leftmargin=*]
    \item More complex model
    \item More features
    \item Reduce regularization
    \item More training time
\end{itemize}

\subsection{High Variance (Overfitting)}

\textbf{Signs:}
\begin{itemize}[leftmargin=*]
    \item Low training error
    \item High test error
    \item Large gap: training vs test
\end{itemize}

\textbf{Solutions:}
\begin{itemize}[leftmargin=*]
    \item Simpler model
    \item More data
    \item Feature selection
    \item Regularization
    \item Early stopping
    \item Ensemble (bagging)
\end{itemize}

\subsection{Complexity Controls by Model}

\begin{tabular}{|p{2cm}|p{2.2cm}|p{2.2cm}|}
\hline
\textbf{Model} & \textbf{$\uparrow$ Complex} & \textbf{$\downarrow$ Complex} \\ \hline
LinReg & Add features & Regularize \\ \hline
kNN & $\downarrow$ k & $\uparrow$ k \\ \hline
Tree & $\uparrow$ depth & $\downarrow$ depth \\ \hline
RF & More trees & Fewer trees \\ \hline
Boosting & More iter & Fewer iter \\ \hline
SVM & $\uparrow$ C & $\downarrow$ C \\ \hline
\end{tabular}

% ============================================================
\section{Common Workflow Mistakes}
% ============================================================

\begin{tipbox}[EXAM GOLD - Spot These Errors!]
These are the most common mistakes tested!
\end{tipbox}

\subsection{1. Data Leakage}

\textbf{WRONG:}
\begin{lstlisting}[language=R]
X_scaled <- scale(X)  # Use all data
train_test_split(X_scaled)
\end{lstlisting}

\textbf{CORRECT:}
\begin{lstlisting}[language=R]
split_data()  # Split first
scaler <- preProcess(train)  # Fit on train
train_scaled <- predict(scaler, train)
test_scaled <- predict(scaler, test)
\end{lstlisting}

\textbf{Common Sources:}
\begin{itemize}[leftmargin=*]
    \item Imputation using full data
    \item Feature selection on full data
    \item Scaling before split
    \item Any transform using global stats
\end{itemize}

\subsection{2. Validation Set Reuse}

\textbf{WRONG:}
\begin{itemize}[leftmargin=*]
    \item Tune many hyperparameters on validation
    \item Report validation accuracy as final
\end{itemize}

\textbf{CORRECT:}
\begin{itemize}[leftmargin=*]
    \item Use validation/CV for tuning
    \item Final eval on held-out test set
    \item Never touch test until final
\end{itemize}

\subsection{3. Feature Selection Bias}

\textbf{WRONG:}
\begin{lstlisting}[language=R]
top_features <- select(X, y)  # All data
train_test_split(X[, top_features])
\end{lstlisting}

\textbf{CORRECT:}
\begin{lstlisting}[language=R]
train_test_split(X)  # Split first
top <- select(X_train, y_train)  # Train only
\end{lstlisting}

\subsection{4. Temporal Leakage}

\textbf{WRONG:}
\begin{itemize}[leftmargin=*]
    \item Randomly shuffle time series
    \item Use future to predict past
\end{itemize}

\textbf{CORRECT:}
\begin{itemize}[leftmargin=*]
    \item Time-based splits
    \item Train on past, test on future
    \item Never shuffle time-dependent data
\end{itemize}

\subsection{5. Overfitting Detection}

\textbf{Red Flags:}
\begin{itemize}[leftmargin=*]
    \item Training acc $\gg$ Test acc
    \item 100 features, 50 observations
    \item Perfect training performance
\end{itemize}

\textbf{Example:}
\begin{lstlisting}
Training Error: 0.02
Test Error: 0.45
-> Clear overfitting!
\end{lstlisting}

\subsection{6. Proper CV with Preprocessing}

\textbf{WRONG:}
\begin{lstlisting}[language=R]
X_pca <- prcomp(X)$x[, 1:5]  # All data
cv_results <- cross_validate(X_pca, y)
\end{lstlisting}

\textbf{CORRECT:}
\begin{lstlisting}[language=R]
for each fold:
  pca <- prcomp(train_fold)  # Train only
  train_pca <- pca$x[, 1:5]
  test_pca <- predict(pca, test_fold)
  # Train and evaluate
\end{lstlisting}

\subsection{7. Imbalanced Data}

\textbf{WRONG:}
\begin{itemize}[leftmargin=*]
    \item Report only accuracy on 99:1 data
    \item Ignore class imbalance
\end{itemize}

\textbf{CORRECT:}
\begin{itemize}[leftmargin=*]
    \item Use F1, AUC, Kappa
    \item Oversample/undersample
    \item Use class weights
    \item Stratified CV
\end{itemize}

\subsection{8. Multiple Testing}

\textbf{WRONG:}
\begin{itemize}[leftmargin=*]
    \item Try 100 models
    \item Report best (p $< 0.05$)
    \item Claim significance
\end{itemize}

\textbf{CORRECT:}
\begin{itemize}[leftmargin=*]
    \item Adjust for multiple comparisons
    \item Use separate validation
    \item Pre-specify hypotheses
    \item Report all attempts
\end{itemize}

% ============================================================
\section{R Code Quick Reference}
% ============================================================

\subsection{Data Splitting}
\begin{lstlisting}[language=R]
library(caret)
set.seed(123)
idx <- createDataPartition(y, p=0.8)
train <- data[idx, ]
test <- data[-idx, ]
\end{lstlisting}

\subsection{Preprocessing}
\begin{lstlisting}[language=R]
# Standardization
preproc <- preProcess(train, 
    method=c("center","scale"))
train_scaled <- predict(preproc, train)
test_scaled <- predict(preproc, test)

# Missing data
preproc <- preProcess(train, 
    method="medianImpute")
\end{lstlisting}

\subsection{Models}
\begin{lstlisting}[language=R]
# Linear Regression
lm(y ~ ., data=train)

# Logistic
glm(y ~ ., data=train, family=binomial)

# Ridge/Lasso
library(glmnet)
glmnet(X, y, alpha=0)  # Ridge
glmnet(X, y, alpha=1)  # Lasso

# LDA/QDA
library(MASS)
lda(y ~ ., data=train)
qda(y ~ ., data=train)

# kNN
library(class)
knn(train[,-1], test[,-1], train$y, k=5)

# SVM
library(e1071)
svm(y ~ ., data=train, kernel="radial")

# Decision Tree
library(rpart)
rpart(y ~ ., data=train)

# Random Forest
library(randomForest)
randomForest(y ~ ., data=train, ntree=500)

# GBM
library(gbm)
gbm(y ~ ., data=train, n.trees=100)

# XGBoost
library(xgboost)
xgboost(data=X, label=y, nrounds=100)
\end{lstlisting}

\subsection{Cross-Validation}
\begin{lstlisting}[language=R]
library(caret)
ctrl <- trainControl(method="cv", number=10)
model <- train(y ~ ., data=train, 
               method="rf", trControl=ctrl)
\end{lstlisting}

\subsection{Metrics}
\begin{lstlisting}[language=R]
# Confusion Matrix
library(caret)
confusionMatrix(pred, actual)

# ROC/AUC
library(pROC)
roc_obj <- roc(actual, pred_probs)
auc(roc_obj)

# Classification
library(MLmetrics)
Accuracy(y_pred, y_true)
Precision(y_pred, y_true)
Recall(y_pred, y_true)
F1_Score(y_pred, y_true)
\end{lstlisting}

\subsection{Clustering \& Dimensionality}
\begin{lstlisting}[language=R]
# K-means
kmeans(X, centers=3, nstart=25)

# Hierarchical
hc <- hclust(dist(X), method="complete")
cutree(hc, k=3)

# PCA
pca <- prcomp(X, center=TRUE, scale=TRUE)

# t-SNE
library(Rtsne)
Rtsne(X, dims=2, perplexity=30)
\end{lstlisting}

% ============================================================
\section{Exam Strategy}
% ============================================================

\subsection{For "Compare \& Contrast"}
\begin{enumerate}[leftmargin=*]
    \item State algorithmic difference
    \item Compare assumptions
    \item Compare flexibility
    \item When to use each
    \item Bias-variance trade-off
    \item Computational considerations
\end{enumerate}

\subsection{For "Workflow Critique"}
\textbf{Look for:}
\begin{itemize}[leftmargin=*]
    \item ❌ Preprocessing before split
    \item ❌ Using test set multiple times
    \item ❌ Feature selection on full data
    \item ❌ Wrong CV implementation
    \item ❌ Ignoring class imbalance
    \item ❌ Wrong metrics
    \item ❌ Temporal leakage
    \item ❌ Overfitting ignored
\end{itemize}

\subsection{For "Classification Eval"}
\begin{enumerate}[leftmargin=*]
    \item Calculate all metrics
    \item Consider class imbalance
    \item Interpret in context
    \item Suggest appropriate metric
    \item Check if performance adequate
\end{enumerate}

\subsection{For "Pseudocode"}
\begin{itemize}[leftmargin=*]
    \item \texttt{<1>} often = n\_simulations
    \item \texttt{<2>} often = random generation
    \item \texttt{<3>} often = condition
    \item \texttt{<4>} often = calculation
    \item Think: "What is this estimating?"
\end{itemize}

\subsection{For "R Output Interpretation"}
\begin{itemize}[leftmargin=*]
    \item Compare metrics appropriately
    \item Consider bias-variance
    \item Look for overfitting
    \item Suggest improvements
    \item Remember: train error $\leq$ test error
\end{itemize}

% ============================================================
\section{Key Formulas to Memorize}
% ============================================================

\begin{formulabox}[Must-Know Metrics]
\textbf{Accuracy} = $\frac{TP + TN}{Total}$ \\[4pt]
\textbf{Precision} = $\frac{TP}{TP + FP}$ \\[4pt]
\textbf{Recall} = $\frac{TP}{TP + FN}$ \\[4pt]
\textbf{F1} = $\frac{2 \times Prec \times Rec}{Prec + Rec}$ \\[4pt]
\textbf{Specificity} = $\frac{TN}{TN + FP}$
\end{formulabox}

\begin{formulabox}[Model Error]
\textbf{Total Error} = Bias$^2$ + Variance + Irreducible \\[4pt]
\textbf{MSE} = $\frac{1}{n}\sum(y_i - \hat{y}_i)^2$ \\[4pt]
\textbf{R²} = $1 - \frac{\sum(y_i - \hat{y}_i)^2}{\sum(y_i - \bar{y})^2}$
\end{formulabox}

\begin{formulabox}[Key Distributions]
\textbf{Logistic:} $P(Y=1|X) = \frac{1}{1+e^{-X\beta}}$ \\[4pt]
\textbf{Normal:} $f(x) = \frac{1}{\sqrt{2\pi\sigma^2}}e^{-\frac{(x-\mu)^2}{2\sigma^2}}$
\end{formulabox}

% ============================================================
\section{Final Tips}
% ============================================================

\begin{tipbox}[Critical Rules]
\textbf{1.} Always split FIRST, then preprocess \\
\textbf{2.} Never touch test set until final \\
\textbf{3.} Imbalanced data: F1/AUC/Kappa \\
\textbf{4.} Standardize for: kNN, SVM, PCA \\
\textbf{5.} CV for selection, not test set \\
\textbf{6.} Overfitting: train $\ll$ test \\
\textbf{7.} PCA for modeling, t-SNE for viz \\
\textbf{8.} Ridge keeps all, Lasso selects \\
\textbf{9.} Boosting $>$ overfitting than bagging \\
\textbf{10.} Context matters for metrics!
\end{tipbox}

\begin{center}
\textbf{\Large{Good Luck! 🍀}}
\end{center}

\end{multicols}

\end{document}
