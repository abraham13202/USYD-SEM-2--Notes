\documentclass[10pt,landscape,a4paper]{article}
\usepackage[utf8]{inputenc}
\usepackage[landscape,margin=0.5cm]{geometry}
\usepackage{multicol}
\usepackage{enumitem}
\usepackage{xcolor}
\usepackage{titlesec}
\usepackage{tabularx}
\usepackage[compact]{titlesec}

% Compact spacing
\setlength{\parindent}{0pt}
\setlength{\parskip}{0pt}
\setlist{noitemsep,topsep=0pt,leftmargin=*}

% Section formatting
\titleformat{\section}{\normalfont\large\bfseries\color{blue!70!black}}{\thesection}{1em}{}
\titleformat{\subsection}{\normalfont\normalsize\bfseries\color{blue!50!black}}{\thesubsection}{1em}{}
\titlespacing*{\section}{0pt}{4pt}{2pt}
\titlespacing*{\subsection}{0pt}{3pt}{1pt}

% Custom commands
\newcommand{\keyword}[1]{\textbf{\color{red!70!black}#1}}
\newcommand{\concept}[1]{\textit{\color{blue!60!black}#1}}

\begin{document}
\begin{multicols}{3}
\setlength{\columnseprule}{0.4pt}

\begin{center}
\Large{\textbf{INFO5992 EXAM CHEATSHEET}} \\
\small{Understanding IT Innovations - Complete Reference}
\end{center}

\section{1. Innovation Types \& Strategies}

\subsection{Disruptive Innovation}
\begin{itemize}
    \item \keyword{Low-End Disruption}: Targets overserved customers with ``good enough'' cheaper solutions → incumbents ignore
    \item \keyword{New-Market Disruption}: Creates new market serving non-consumers → different value proposition
    \item \keyword{Sustaining Innovation}: Improves existing products along established dimensions → incumbents excel
\end{itemize}

\subsection{Innovation Dilemma}
\begin{itemize}
    \item Incumbents allocate resources to high-margin customers
    \item Ignore low-end threats until too late
    \item Existing capabilities become obsolete
\end{itemize}

\subsection{Competence Types}
\begin{itemize}
    \item \keyword{Competence-Enhancing}: Builds on existing knowledge → incumbents have advantage
    \item \keyword{Competence-Destroying}: Requires new capabilities → startups have advantage (clean slate)
\end{itemize}

\subsection{Technological Discontinuity}
\keyword{Definition}: Dramatic breakthrough that makes existing tech obsolete

\textbf{Why Companies Choose It:}
\begin{itemize}
    \item Overcome competitive stagnation
    \item Access new markets/capabilities
    \item Leapfrog competitors
    \item Respond to market shifts
\end{itemize}

\textbf{Types:}
\begin{itemize}
    \item \concept{Competence-enhancing}: Leverages existing skills
    \item \concept{Competence-destroying}: Requires new expertise
\end{itemize}

\textbf{Impact on Startups:}
\begin{itemize}
    \item Opportunity window for new entrants
    \item Incumbents' advantages neutralized
    \item Market uncertainty → experimentation phase
\end{itemize}

\section{2. Dominant Design \& Lifecycle}

\subsection{Dominant Design Phases}
\begin{enumerate}
    \item \keyword{Fluid Phase}: Multiple competing designs, high uncertainty, product performance focus
    \item \keyword{Transitional Phase}: Design standards emerge, competition intensifies
    \item \keyword{Specific Phase}: One design dominates, shift to process innovation \& cost reduction
\end{enumerate}

\subsection{Why Dominant Design Matters}
\begin{itemize}
    \item Reduces uncertainty for customers \& producers
    \item Enables economies of scale \& network effects
    \item Compatibility with ecosystem/infrastructure
    \item Signals stability to mainstream market
\end{itemize}

\subsection{When Dominant Design May NOT Emerge}
\begin{itemize}
    \item High customization needs (different segments need different designs)
    \item Rapid technological change prevents standardization
    \item Strong network effects lock in multiple competing standards
    \item Regulatory fragmentation across markets
    \item Low switching costs → continuous experimentation
\end{itemize}

\subsection{Choosing Dominant Design (Startup)}
\begin{itemize}
    \item Customer acceptance \& reduced adoption risk
    \item Compatibility with existing systems
    \item Cost \& scalability for mass market
    \item Performance reliability benchmarks
    \item Network effects \& ecosystem leverage
\end{itemize}

\section{3. Technology Adoption Lifecycle}

\subsection{Adoption Segments}
\begin{enumerate}
    \item \keyword{Innovators} (2.5\%): Tech enthusiasts, high risk tolerance
    \item \keyword{Early Adopters} (13.5\%): Visionaries, willing to experiment
    \item \keyword{Early Majority} (34\%): Pragmatists, need proven solutions
    \item \keyword{Late Majority} (34\%): Conservatives, adopt when necessary
    \item \keyword{Laggards} (16\%): Skeptics, resist change
\end{enumerate}

\subsection{Crossing the Chasm}
\begin{itemize}
    \item \keyword{The Gap}: Between Early Adopters \& Early Majority
    \item \keyword{Why Hard}: Early majority demands reliability, references, established standards; startups lack credibility
    \item \keyword{How Dominant Design Helps}: Reduces uncertainty, signals stability, builds trust, enables ecosystem
\end{itemize}

\section{4. Technology Hype Cycle}

\subsection{5 Stages}
\begin{enumerate}
    \item \keyword{Innovation Trigger}: Breakthrough sparks interest
    \item \keyword{Peak of Inflated Expectations}: Unrealistic hype, many experiments
    \item \keyword{Trough of Disillusionment}: Reality fails expectations, failures occur
    \item \keyword{Slope of Enlightenment}: Practical applications emerge, \concept{dominant design often emerges here}
    \item \keyword{Plateau of Productivity}: Mainstream adoption, standards solidify
\end{enumerate}

\textbf{Relationship with Dominant Design:} Competing designs in early stages → dominant design stabilizes market during Slope/Plateau → reduces uncertainty → enables scale

\section{5. Value Chain vs Value Network}

\subsection{Value Chain (Porter)}
\begin{itemize}
    \item \keyword{Linear model}: Sequential activities adding value
    \item \concept{Primary activities}: Inbound logistics, operations, outbound logistics, marketing/sales, service
    \item \concept{Support activities}: Infrastructure, HR, tech development, procurement
    \item \textbf{Focus}: Internal efficiency, cost reduction
    \item \textbf{Example}: Manufacturing (raw materials → production → distribution → retail)
\end{itemize}

\subsection{Value Network}
\begin{itemize}
    \item \keyword{Network model}: Multiple players co-create value
    \item Interdependent relationships, not linear
    \item Value created through interactions \& connections
    \item \textbf{Focus}: Ecosystem collaboration, network effects
    \item \textbf{Example}: Platform (app developers + users + payment providers all create value together)
\end{itemize}

\subsection{Key Differences}
\begin{tabular}{|l|l|}
\hline
\textbf{Value Chain} & \textbf{Value Network} \\
\hline
Linear flow & Multi-directional \\
Single firm focus & Ecosystem focus \\
Sequential activities & Simultaneous interactions \\
Efficiency-driven & Network effects-driven \\
\hline
\end{tabular}

\section{6. Value Proposition Canvas}

\subsection{Customer Profile}
\begin{itemize}
    \item \keyword{Jobs}: Tasks customers accomplish (functional, social, emotional)
    \item \keyword{Pains}: Obstacles, risks, negative emotions
    \item \keyword{Gains}: Desired outcomes, benefits, aspirations
\end{itemize}

\subsection{Value Map}
\begin{itemize}
    \item \keyword{Products/Services}: What you offer
    \item \keyword{Pain Relievers}: How you eliminate/reduce pains
    \item \keyword{Gain Creators}: How you create benefits
\end{itemize}

\subsection{Achieving Fit}
\begin{itemize}
    \item Align value map with customer profile
    \item Address important jobs, alleviate major pains, create meaningful gains
    \item = \keyword{Product-Market Fit (PMF)}
\end{itemize}

\subsection{Value Types}
\begin{itemize}
    \item \concept{Functional}: Solves practical problem
    \item \concept{Emotional}: Makes customer feel better
    \item \concept{Social}: Status, belonging, impression
    \item \concept{Life-Changing}: Transforms customer's life
\end{itemize}

\section{7. Customer Development Process}

\subsection{4 Stages}
\begin{enumerate}
    \item \keyword{Customer Discovery}: Test problem hypotheses, identify real pains/needs, get out of building
    \item \keyword{Customer Validation}: Test solution with MVP, achieve repeatable sales model
    \item \keyword{Customer Creation}: Scale demand, execute marketing/sales
    \item \keyword{Company Building}: Transition to execution-focused organization
\end{enumerate}

\subsection{Key Principles}
\begin{itemize}
    \item \keyword{Get Out of Building}: Talk to real customers, not desk research
    \item \keyword{Build-Measure-Learn}: Fast iteration cycles
    \item \keyword{Pivot or Persevere}: Change strategy based on validated learning
    \item \keyword{MVP}: Test assumptions with least effort
\end{itemize}

\subsection{Creating Value in Customer Dev}
\begin{itemize}
    \item Align value map with customer profile through interviews
    \item Validate problem-solution fit before building
    \item Iterate based on feedback to strengthen value proposition
    \item Test with prototypes/wireframes in discovery phase
    \item Build high-fidelity MVP in validation phase
    \item Verify scalability \& repeatability of business model
\end{itemize}

\section{8. Lean Startup vs Waterfall}

\subsection{Why Waterfall Fails for Startups}
\begin{itemize}
    \item Linear \& rigid → can't adapt to feedback
    \item Late customer involvement → build wrong product
    \item High waste if assumptions wrong
    \item Assumes clear requirements upfront (unrealistic for innovation)
\end{itemize}

\subsection{Lean Advantages}
\begin{itemize}
    \item \keyword{Build-Measure-Learn loop}: Fast prototyping \& testing
    \item Early customer involvement → validate assumptions
    \item Flexible pivots based on feedback
    \item Resource-efficient → focus on essential features
    \item Reduces risk through validated learning
    \item Supports experimentation \& iteration
\end{itemize}

\section{9. Open Innovation}

\subsection{3 Innovation Flows}
\begin{enumerate}
    \item \keyword{Outside-In}: Bring external ideas/tech into company (licensing in, partnerships, crowdsourcing)
    \item \keyword{Inside-Out}: Commercialize internal ideas externally (spin-offs, licensing out, selling IP)
    \item \keyword{Coupled}: Combine both through alliances, joint ventures, co-creation
\end{enumerate}

\subsection{Open Source vs Proprietary}

\begin{tabular}{|p{3.5cm}|p{3.5cm}|}
\hline
\textbf{Open Source} & \textbf{Proprietary} \\
\hline
Community-driven innovation & Controlled development \\
Lower costs (no licensing) & Revenue from licenses \\
Transparency \& auditability & Trade secrets protected \\
Faster iteration & Competitive advantage \\
Interoperability \& standards & Lock-in strategy \\
\hline
\end{tabular}

\subsection{Why Open Source Drives Innovation (4+ Reasons)}
\begin{enumerate}
    \item \keyword{Faster innovation}: Global community contributes improvements continuously
    \item \keyword{Lower costs}: No licensing fees → allocate resources to differentiation
    \item \keyword{Transparency}: Code auditability builds trust (critical for finance, healthcare)
    \item \keyword{Interoperability}: Easier integration with existing systems via standards
    \item \keyword{Flexibility}: Customize to specific needs without vendor lock-in
    \item \keyword{Community support}: Faster bug fixes, security patches, feature development
    \item \keyword{Talent attraction}: Developers prefer working with open technologies
\end{enumerate}

\subsection{Open Source Benefits for Startups}
\begin{itemize}
    \item \concept{Standardization}: Compliance with industry norms (e.g., Linux, PostgreSQL)
    \item \concept{Interoperability}: Easy integration with existing systems
    \item \concept{Community innovation}: Continuous improvements (security, performance)
    \item \concept{Lower costs}: Allocate resources to core differentiation
    \item \concept{Trust}: Code auditability for regulatory compliance
\end{itemize}

\subsection{Closed to Open Innovation Shift}

\textbf{Reasons for Shift:}
\begin{itemize}
    \item Access external expertise \& reduce R\&D costs
    \item Faster time-to-market through collaboration
    \item Tap into global talent pool
    \item Share development risk with ecosystem
\end{itemize}

\textbf{Risks:}
\begin{itemize}
    \item Loss of IP control \& competitive advantage
    \item Quality control challenges with external contributors
    \item Coordination overhead \& integration complexity
    \item Risk of free-riding by competitors
\end{itemize}

\section{10. APIs \& Modularity}

\subsection{API Types}
\begin{enumerate}
    \item \keyword{API as Product}: Core business offering (Stripe, Twilio, AWS)
    \item \keyword{API Enhancing}: Adds functionality to existing products (Google Maps API)
    \item \keyword{API Promoting}: Drives adoption \& ecosystem growth (Twitter API)
\end{enumerate}

\subsection{How APIs Enable Innovation}
\begin{itemize}
    \item \concept{Modularity}: Decouple components → independent development
    \item \concept{Scalability}: Add features without rebuilding core system
    \item \concept{Third-party integration}: External developers extend functionality
    \item \concept{Ecosystem growth}: Network effects as more developers build on platform
    \item \concept{Faster iteration}: Update modules independently
\end{itemize}

\subsection{Modularity in Software Architecture}
\begin{itemize}
    \item \keyword{Definition}: Breaking system into independent, interchangeable components
    \item \keyword{Benefits}:
    \begin{itemize}
        \item Parallel development by different teams
        \item Easier testing \& debugging
        \item Component reusability across projects
        \item Lower coupling → changes don't cascade
        \item Enable third-party innovation via APIs
    \end{itemize}
    \item \keyword{Example}: Microservices architecture (authentication, payment, notification as separate modules)
\end{itemize}

\section{11. Platform Business Models}

\subsection{Platform Ecosystem Model}
\begin{itemize}
    \item \keyword{Integrator Platform}: Mediates between external innovators \& customers
    \item Controls transactions \& interactions
    \item Benefits from network effects
    \item Monetizes through fees, subscriptions, data
\end{itemize}

\subsection{Key Players in Platform}
\begin{itemize}
    \item \concept{Producers}: Create value (app developers, content creators)
    \item \concept{Consumers}: Use value (end users)
    \item \concept{Platform Owner}: Provides infrastructure \& governance
    \item \concept{Complementors}: Enhance platform value (device makers, payment providers)
    \item \concept{Regulators}: Ensure compliance \& safety
\end{itemize}

\subsection{Platform Types}
\begin{enumerate}
    \item \keyword{Transaction platforms}: Facilitate exchanges (Uber, Airbnb, eBay)
    \item \keyword{Innovation platforms}: Enable third-party development (iOS, Android, AWS)
    \item \keyword{Integrated platforms}: Combine both (Apple ecosystem)
\end{enumerate}

\subsection{Platform Monetization}
\begin{itemize}
    \item Transaction fees (commission on each sale)
    \item Subscription fees (premium features)
    \item Advertising revenue (targeted ads)
    \item Freemium model (free basic, paid premium)
    \item Data monetization (insights, analytics)
\end{itemize}

\section{12. Crowdsourcing}

\subsection{4 Types}
\begin{enumerate}
    \item \keyword{Knowledge Discovery}: Tap distributed expertise (InnoCentive - scientific problems)
    \item \keyword{Broadcast Search}: Post problems, best solution wins (Kaggle - data science competitions)
    \item \keyword{Peer-Vetted Creative}: Community evaluates ideas (Threadless - t-shirt designs)
    \item \keyword{Distributed Human Intelligence Tasks}: Micro-tasks at scale (Amazon MTurk - labeling, surveys)
\end{enumerate}

\subsection{Challenges}
\begin{itemize}
    \item Quality control \& reliability of contributions
    \item Intellectual property ownership disputes
    \item Participant motivation \& retention
    \item Coordination costs \& management overhead
    \item Free-riding \& unequal contribution
\end{itemize}

\subsection{Benefits}
\begin{itemize}
    \item Access to diverse expertise globally
    \item Cost-effective compared to in-house R\&D
    \item Faster problem-solving through parallel efforts
    \item Identify innovative solutions from unexpected sources
\end{itemize}

\section{13. Lead Users}

\subsection{Who They Are}
\begin{itemize}
    \item Experience needs ahead of market (future-oriented)
    \item High benefit from solutions (strong motivation)
    \item Often innovate solutions themselves (proactive)
\end{itemize}

\subsection{Why Valuable for Sustainable Innovation}
\begin{itemize}
    \item Provide insights into future market needs
    \item Co-create products \& provide early feedback
    \item Accelerate adoption as opinion leaders
    \item Reduce uncertainty \& validate design for scalability
    \item Test extreme use cases that normal users won't encounter
\end{itemize}

\subsection{Why Normal Users Less Effective (5 Reasons)}
\begin{enumerate}
    \item \keyword{Lack vision for future needs}: Focus on immediate problems, not emerging trends
    \item \keyword{Limited technical expertise}: Can't suggest advanced/disruptive solutions → incremental feedback only
    \item \keyword{Lower risk appetite}: Prefer proven solutions, avoid untested tech → slow radical innovation
    \item \keyword{Incremental feedback only}: Minor improvements, not breakthrough ideas
    \item \keyword{Biased by dominant designs}: Accustomed to current standards → resist creative thinking
\end{enumerate}

\section{14. Knowledge Sharing in Startups}

\subsection{How It Drives Innovation}
\begin{itemize}
    \item \concept{Cross-pollination of ideas}: Different perspectives spark creativity
    \item \concept{Faster problem-solving}: Collective intelligence reduces bottlenecks
    \item \concept{Reduced duplication}: Share learnings to avoid repeated mistakes
    \item \concept{Skill development}: Team learns from each other
    \item \concept{Organizational learning}: Capture \& reuse tacit knowledge
\end{itemize}

\subsection{Mechanisms/Practices for Knowledge Sharing}
\begin{enumerate}
    \item \keyword{Cross-functional teams}: Break silos, share domain expertise
    \item \keyword{Regular knowledge-sharing sessions}: Sprint demos, brown bags, tech talks
    \item \keyword{Documentation culture}: Wikis, runbooks, design docs
    \item \keyword{Mentorship \& pair programming}: Tacit knowledge transfer
    \item \keyword{Retrospectives}: Capture lessons learned from projects
    \item \keyword{Internal communities of practice}: Groups focused on specific domains
    \item \keyword{Open communication tools}: Slack channels, forums for async sharing
\end{enumerate}

\section{15. Coopetition}

\subsection{Definition}
Collaboration with competitors instead of pure disruption

\subsection{Why Startups Choose It}
\begin{itemize}
    \item \keyword{High entry barriers}: Costly infrastructure, regulations → partnerships needed
    \item \keyword{Shared resources}: Access to distribution, tech, customer base
    \item \keyword{Network effects}: Faster adoption \& ecosystem growth through collaboration
    \item \keyword{Risk reduction}: Lower financial \& market uncertainty
    \item \keyword{Complementary strengths}: Startup innovation + incumbent scale/credibility
\end{itemize}

\subsection{Examples}
\begin{itemize}
    \item Fintech startups partnering with banks (access to customers + regulatory compliance)
    \item EV startups using established automakers' charging networks
    \item SaaS companies integrating with competitors to serve customers better
\end{itemize}

\section{16. Organizational Structures}

\subsection{Mechanistic Structure}
\begin{itemize}
    \item High formalization, centralized decisions
    \item Rigid hierarchy, clearly defined roles
    \item Efficient for stability, routine tasks
    \item Limits creativity \& flexibility
\end{itemize}

\subsection{Organic Structure}
\begin{itemize}
    \item Low formalization, decentralized decisions
    \item Flexible processes, collaborative
    \item Encourages adaptability \& innovation
\end{itemize}

\subsection{Making Mechanistic Innovative}
\begin{enumerate}
    \item \keyword{Cross-functional teams}: Encourage collaboration across departments
    \item \keyword{Reduce formalization}: Allow flexibility in processes
    \item \keyword{Promote risk-taking}: Failure viewed as learning, not punishment
    \item \keyword{Allocate innovation time}: e.g., Google's ``20\% time''
    \item \keyword{Reward creativity}: Incentives for innovative ideas
    \item \keyword{Skunk Works teams}: Small, autonomous groups for disruptive projects
    \item \keyword{Flatten hierarchy}: Reduce approval layers for faster decisions
\end{enumerate}

\section{17. Startup Failure Factors}

\subsection{Top 3 Reasons}
\begin{enumerate}
    \item \keyword{Lack of Product-Market Fit}: Product doesn't solve real customer needs → poor adoption. Built without validating demand.
    \item \keyword{Insufficient Capital}: Run out of cash before achieving scale. High burn rates, long R\&D cycles (hardware/deep-tech).
    \item \keyword{Weak Business Model}: No clear revenue model, mispricing, unclear monetization → unsustainable operations.
\end{enumerate}

\subsection{Other Contributing Factors}
\begin{itemize}
    \item Poor team dynamics \& founder conflicts
    \item Inability to pivot when assumptions fail
    \item Intense competitive pressure
    \item Premature scaling before PMF
    \item Ignoring customer feedback
\end{itemize}

\section{18. Business Model Canvas}

\subsection{Key Components}

\textbf{Customer Segments:} Who are you serving?
\begin{itemize}
    \item Mass market, niche, segmented, diversified, multi-sided
\end{itemize}

\textbf{Value Propositions:} What value do you deliver?
\begin{itemize}
    \item Newness, performance, customization, design, brand, price, convenience, risk reduction
\end{itemize}

\textbf{Key Resources:} What assets required?
\begin{itemize}
    \item \concept{Tangible}: Physical (facilities, equipment), Financial (cash, credit)
    \item \concept{Intangible}: Intellectual (patents, IP, data), Human (skills, expertise, talent)
\end{itemize}

\textbf{Key Activities:} What do you do?
\begin{itemize}
    \item Production, problem-solving, platform/network management
\end{itemize}

\textbf{Revenue Streams:} How do you make money?
\begin{itemize}
    \item Asset sale, usage fee, subscription, licensing, advertising, freemium
\end{itemize}

\subsection{Why Customer Segments \& Value Props Critical}
\begin{enumerate}
    \item \keyword{Focus resources}: Target right customers, avoid wasting effort
    \item \keyword{Achieve PMF}: Clear value prop aligned with segment needs
    \item \keyword{Differentiation}: Stand out from competitors in specific segments
    \item \keyword{Scalability}: Repeatable model for similar customers
    \item \keyword{Revenue model}: Pricing \& monetization fit customer willingness to pay
\end{enumerate}

\section{19. Diffusion of Innovation}

\subsection{Diffusion Strategies for Tech Startups}
\begin{enumerate}
    \item \keyword{Target early adopters}: Identify visionaries willing to take risks, use as reference customers
    \item \keyword{Build credibility}: Case studies, testimonials, proof of concept with reputable customers
    \item \keyword{Reduce adoption barriers}: Free trials, freemium model, easy onboarding
    \item \keyword{Leverage network effects}: Incentivize referrals, viral loops, community building
    \item \keyword{Partner with complementors}: Integrate with established platforms/ecosystems
    \item \keyword{Educate market}: Content marketing, webinars, thought leadership to build awareness
    \item \keyword{Align with dominant design}: Reduce perceived risk for mainstream adopters
\end{enumerate}

\subsection{Barriers to Innovation Implementation}
\begin{enumerate}
    \item \keyword{Organizational inertia}: Resistance to change, ``not invented here'' syndrome, fear of disrupting existing business
    \item \keyword{Resource constraints}: Insufficient budget, talent shortage, competing priorities
    \item \keyword{Lack of leadership support}: Innovation not prioritized, risk-averse culture
    \item \keyword{Poor communication}: Silos prevent knowledge sharing, misaligned incentives
\end{enumerate}

\section{20. Agentic AI vs Autonomous}

\subsection{Agentic AI}
\begin{itemize}
    \item \keyword{Proactive \& goal-driven}: Pursues objectives independently
    \item \keyword{Adaptive reasoning}: Adjusts strategy based on environment
    \item \keyword{Multi-step planning}: Breaks down complex tasks
    \item \textbf{Example}: AI research assistant formulating hypotheses \& designing experiments
\end{itemize}

\subsection{Autonomous Systems}
\begin{itemize}
    \item \keyword{Reactive \& task-specific}: Executes predefined tasks
    \item \keyword{Rule-based}: Follows programmed instructions
    \item \keyword{Limited adaptability}: Within narrow scope
    \item \textbf{Example}: Self-driving car following traffic rules
\end{itemize}

\section{21. General Purpose Technology}

\subsection{Characteristics of GPT}
\begin{itemize}
    \item Wide applicability across sectors
    \item Continuous improvement over time
    \item Spawns complementary innovations
    \item Transforms entire economies
    \item \textbf{Examples}: Electricity, internet, steam engine
\end{itemize}

\subsection{Can Generative AI be GPT?}

\textbf{YES - Supporting Arguments:}
\begin{enumerate}
    \item \keyword{Wide applicability}: Content creation, coding, design, research, customer service across all industries
    \item \keyword{Continuous improvement}: Rapid iteration, models improving exponentially
    \item \keyword{Complementary innovations}: New tools, applications, business models emerging
    \item \keyword{Productivity transformation}: Automating knowledge work at scale
\end{enumerate}

\textbf{Counterarguments (if needed):}
\begin{itemize}
    \item Still early stage, adoption not yet universal
    \item Economic transformation impact unclear
\end{itemize}

\section{22. EV Types (if relevant)}

\subsection{Battery Electric Vehicle (BEV)}
\begin{itemize}
    \item 100\% electric, no combustion engine
    \item Zero tailpipe emissions
    \item \textbf{Example}: Tesla Model 3, Nissan Leaf
\end{itemize}

\subsection{Plug-in Hybrid (PHEV)}
\begin{itemize}
    \item Electric motor + combustion engine
    \item Can run on electric only (limited range), then switches to gas
    \item \textbf{Example}: Toyota Prius Prime, Chevrolet Volt
\end{itemize}

\subsection{Range-Extended EV (REEV)}
\begin{itemize}
    \item Primarily electric, small engine as generator (doesn't drive wheels)
    \item Extends range when battery depleted
    \item \textbf{Example}: BMW i3 with range extender
\end{itemize}

\section{23. Exam Answer Templates}

\subsection{2-Mark Question (2-3 lines)}
\begin{itemize}
    \item Define concept in 1 line
    \item Key reason/application in 1-2 lines
    \item \textbf{Example}: ``PMF is degree to which product satisfies market demand. Value Prop Canvas achieves it by aligning customer jobs/pains/gains with pain relievers \& gain creators.''
\end{itemize}

\subsection{5-Mark Question}
\begin{itemize}
    \item 2-3 key points with brief explanation
    \item Each point: 1-2 lines
    \item Use bullet format
    \item \textbf{Example}: See Section 1 questions in answer key
\end{itemize}

\subsection{10-Mark Scenario}
\begin{itemize}
    \item Apply framework clearly
    \item 4-5 bullet points
    \item Each point: claim + reasoning + scenario application
    \item Use keywords marker looks for
    \item \textbf{Example}: See Section 2 \& 3 questions in answer key
\end{itemize}

\section{24. Key Exam Keywords}

\subsection{Innovation Type}
disruptive, sustaining, low-end, new-market, competence-destroying, competence-enhancing, radical, incremental

\subsection{Adoption}
chasm, early adopters, mainstream, pragmatists, proven solution, references, credibility

\subsection{Design}
dominant design, standardization, compatibility, network effects, economies of scale, interoperability

\subsection{Customer Dev}
MVP, pivot, validated learning, get out of building, problem-solution fit, PMF, iteration

\subsection{Value Prop}
jobs-pains-gains, pain relievers, gain creators, functional/emotional/social value, fit

\subsection{Lean}
Build-Measure-Learn, iterate, waste reduction, customer feedback loops, fast experimentation

\subsection{Open Innovation}
outside-in, inside-out, coupled, ecosystem, interoperability, collaboration

\subsection{Platform}
network effects, multi-sided, transaction fees, ecosystem, complementors

\section{25. Quick Reference Frameworks}

\subsection{When Analyzing Innovation Type}
\begin{enumerate}
    \item Does it target overserved customers with cheaper solution? → \concept{Low-end disruption}
    \item Does it create new market for non-consumers? → \concept{New-market disruption}
    \item Does it improve along existing dimensions? → \concept{Sustaining}
    \item Does it make existing capabilities obsolete? → \concept{Competence-destroying}
\end{enumerate}

\subsection{When Analyzing Platform Ecosystem}
\begin{enumerate}
    \item Identify key players (producers, consumers, complementors)
    \item Show mediation role of platform
    \item Explain network effects
    \item Discuss monetization strategy
\end{enumerate}

\subsection{When Analyzing Customer Development}
\begin{enumerate}
    \item Discovery: What problem? Get out of building, test hypotheses
    \item Validation: Does solution work? MVP, test sell, positioning
    \item Pivot considerations: What's wrong? How to address?
    \item Use Build-Measure-Learn terminology
\end{enumerate}

\subsection{When Analyzing Value Proposition}
\begin{enumerate}
    \item Customer jobs (what trying to do?)
    \item Pains (what frustrations?)
    \item Gains (what benefits desired?)
    \item How product addresses each
\end{enumerate}

\section{26. Common Question Patterns}

\subsection{``What type of innovation is X?''}
\begin{itemize}
    \item State innovation type clearly
    \item Provide 2 reasons with explanation
    \item Reference scenario specifics
\end{itemize}

\subsection{``How can X help achieve Y?''}
\begin{itemize}
    \item Briefly define X
    \item List mechanisms/ways (3-4 points)
    \item Connect to outcome Y
\end{itemize}

\subsection{``Why do startups fail / choose coopetition / etc?''}
\begin{itemize}
    \item List 3-5 clear reasons
    \item Each reason: 1-2 line explanation
    \item Use scenario context if given
\end{itemize}

\subsection{``Explain difference between X and Y''}
\begin{itemize}
    \item Define both briefly
    \item Highlight 3-4 key differences
    \item Use table if helpful
    \item Provide example for each
\end{itemize}

\end{multicols}
\end{document}
